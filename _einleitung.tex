\chapter{Einleitung}

Vor einiger Zeit wurde Software primär mit dem Hintergrund entwickelt, später auf einem einzelnen Rechner zu laufen, sodass man nicht mit vielen Seiteneffekten durch eine andere Konfiguration auf Software- oder Hardware-Ebene zu rechnen hatte. Heute arbeiten viele Entwickler gleichzeitig auf verschiedensten Geräten an einer einzelnen Webanwendung oder einem Webservice. Es gibt eine unendliche Anzahl von Konfigurationen durch verschiedene Hardware-Konfigurationen, verschiedene Betriebssysteme oder auch installierte Programme und Dienste. All diese Unterschiede können die entwickelte Anwendung bereits im Entwicklungsstadium auf unterschiedliche Weise beeinflussen und für unterschiedliches Verhalten sorgen. Wird die Anwendung später ausgerollt und auf einem Server ausgeführt, muss sie wiederum in einer anderen Umgebung arbeiten. Das Ziel ist es deshalb die Software nicht von der Plattform auf der sie ausgeführt wird abhängig zu machen und sie immer in der gleichen Umgebung laufen zu lassen. Damit kann die Software auf beliebig vielen Systemen ausgeführt werden mit gleichzeitiger Sicherheit, dass sie sich auf allen Systemen gleich verhalten wird. Um dies zu ermöglichen, kann die Software entweder in einer virtuellen Maschine, einem simulierten Rechner, oder einem Container, einer leichtgewichtigeren Form der Virtualisierung, ausgeführt werden. Im Folgenden werde ich auf die Arbeitsweise von Containern im Vergleich zu virtuellen Maschinen eingehen und im speziellen Docker, einer Software für eine einfachere Handhabung von Containern, vorstellen.
